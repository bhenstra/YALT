\chapter{Introduction}

In this document, we will explore the fundamental principles of LaTeX and its application in creating structured, high-quality documents. The goal of this template is to demonstrate the basic features of LaTeX while offering a foundation for professional document formatting. Throughout the document, we will focus on clarity, readability, and efficiency.

Additionally, the structure of the document has been designed to accommodate a variety of content, including text, mathematical expressions, and graphics. By following this structure, users can easily adapt the template for their own needs, whether for academic, professional, or personal projects.

This introductory section will provide an overview of the document’s structure and the purpose of the content that follows. We will cover the essentials first, followed by more advanced topics that may require deeper understanding or specialized knowledge.

The quick brown fox jumps over the lazy dog. This is a pangram used to test the typing and font rendering on various devices. It contains all the letters of the English alphabet, making it ideal for testing text layout, font rendering, and hyphenation patterns.

Lorem ipsum dolor sit amet, consectetur adipiscing elit. Quisque sit amet accumsan arcu. Sed hendrerit suscipit velit, a sollicitudin nisi tempus ac. Suspendisse in leo sem. Ut ut mi id ante efficitur gravida. Integer sed suscipit velit. Nam non dolor vel ipsum aliquet varius. Vivamus mollis ante in augue cursus fermentum. Aliquam erat volutpat. Nulla sed velit erat. Nulla a condimentum sapien, et vehicula libero. Sed tincidunt scelerisque ligula, ac pharetra nisi aliquet eget.

Sed vulputate risus vitae ante tincidunt vehicula. Curabitur id scelerisque felis. Mauris gravida, sapien at eleifend condimentum, libero magna convallis eros, nec pretium lorem ligula nec lorem. Quisque in nunc ac urna feugiat gravida. Nunc a mi vitae dui tincidunt consectetur. Nulla gravida mauris vitae tristique gravida. Vestibulum condimentum ligula nisl, sit amet porttitor lorem porttitor vel.

Praesent vitae gravida mi. In non felis metus. Duis at nisi eget erat convallis tempus ut ut risus. Mauris vehicula justo ac sapien placerat, a tincidunt metus feugiat. Nullam pretium mollis dui non facilisis. Cras sit amet tortor eu erat sodales tincidunt. Aliquam erat volutpat. Nulla facilisi. Fusce a mi et ligula interdum malesuada vel eu lorem.

Cras convallis, risus id hendrerit suscipit, magna velit tristique augue, eget mollis turpis enim sit amet orci. Suspendisse sit amet auctor nulla. Etiam iaculis risus orci, ac blandit ante hendrerit et. Integer viverra, odio nec posuere tincidunt, odio ligula tempor eros, id mollis nulla elit non felis. Pellentesque fermentum ex risus, sed maximus ligula tristique in.

Nulla facilisi. Nam ac ante id ipsum tincidunt elementum. Cras ultrices consectetur augue, ac vulputate nulla malesuada id. Fusce tempor metus vel libero rhoncus, non interdum enim bibendum. Etiam laoreet justo ut tortor tempor, non tristique felis tincidunt. Ut eget orci eu nulla lobortis auctor ut ut est. Ut ac elit sit amet nisi interdum interdum. Quisque non vulputate nunc, eu sollicitudin arcu. Integer fringilla, dui id laoreet vulputate, felis sapien iaculis purus, eu viverra nunc purus sit amet erat.



\newpage

\chapter{LaTeX}

\section{History}
LaTeX is a typesetting system widely used for producing high-quality documents, particularly in the fields of academia, science, and engineering. It was originally developed by Leslie Lamport in the early 1980s as a high-level interface for the TeX typesetting system, which itself was created by Donald Knuth in the late 1970s. TeX, the foundation of LaTeX, was designed to provide precise control over document layout and typographic quality.

\begin{figure}[h!]
	\centering
	\includegraphics[width=0.6\textwidth]{sources/FrontImg.png} % Breedte op 60% van tekstbreedte
	\caption{An example of an image.}
	\label{fig:frontimage}
\end{figure}

TeX was developed by Donald Knuth as part of his project to improve the typesetting of his own books, particularly *The Art of Computer Programming*. Knuth sought to create a typesetting system that could produce professional-quality printed works without relying on proprietary software or expensive typesetting machines. The goal was to provide an open-source alternative to the traditional typesetting systems used in academia.

Lamport's LaTeX, built on top of TeX, aimed to simplify the user experience by providing macros and predefined styles for common document elements such as articles, reports, and books. This allowed users to focus on the content of their documents while LaTeX handled the layout and formatting automatically. As a result, LaTeX quickly became popular in academic circles, particularly for documents that contained complex mathematical formulas, as TeX’s precise mathematical typesetting capabilities were unmatched at the time.

One of the core features of LaTeX is its emphasis on separating content from presentation. Instead of focusing on the layout as one might in word processors, users write LaTeX documents using plain text with markup commands for structure and formatting. This makes it ideal for collaborative projects, as the structure of the document is separate from its appearance. LaTeX documents can be easily shared, and the formatting will remain consistent across different platforms.

LaTeX’s extensibility is another key reason for its success. Over the years, many packages have been developed that extend its functionality, allowing users to add advanced features such as bibliographies, citation management, table of contents, cross-referencing, and more. These packages make LaTeX a highly flexible and powerful tool for a wide variety of document types.

Despite its steep learning curve for new users, LaTeX continues to be a preferred choice for those who require precision and high-quality output. It is especially favored by mathematicians, engineers, and researchers for its ability to handle complex formulas, cross-referencing, and the production of professional-level documents.

\section{Equation}

Here is the famous equation in inline mode: $E = mc^2$.

And here it is displayed on its own line:
\[
E = mc^2
\]

"The equation $E = mc^2$ expresses a profound truth about the relationship between mass and energy. It tells us that mass can be converted into energy and vice versa. This insight arose from the theory of special relativity, and it fundamentally altered our understanding of the universe. I never claimed to have invented the equation; it was simply a natural consequence of the laws of physics as I understood them."

"What is remarkable about this equation is not just its simplicity, but its far-reaching implications. It led to the development of technologies like nuclear energy, and it continues to be a cornerstone of modern physics."

"However, while $E = mc^2$ is often quoted as my 'most famous' equation, it represents just a small part of a much deeper theory. The full beauty of relativity is captured in its more complex mathematical formulations, where space and time themselves are intertwined in ways that were once unimaginable."

\section{Some text}
The quick brown fox jumps over the lazy dog. This is a pangram used to test the typing and font rendering on various devices. It contains all the letters of the English alphabet, making it ideal for testing text layout, font rendering, and hyphenation patterns.

Lorem ipsum dolor sit amet, consectetur adipiscing elit. Quisque sit amet accumsan arcu. Sed hendrerit suscipit velit, a sollicitudin nisi tempus ac. Suspendisse in leo sem. Ut ut mi id ante efficitur gravida. Integer sed suscipit velit. Nam non dolor vel ipsum aliquet varius. Vivamus mollis ante in augue cursus fermentum. Aliquam erat volutpat. Nulla sed velit erat. Nulla a condimentum sapien, et vehicula libero. Sed tincidunt scelerisque ligula, ac pharetra nisi aliquet eget.

Sed vulputate risus vitae ante tincidunt vehicula. Curabitur id scelerisque felis. Mauris gravida, sapien at eleifend condimentum, libero magna convallis eros, nec pretium lorem ligula nec lorem. Quisque in nunc ac urna feugiat gravida. Nunc a mi vitae dui tincidunt consectetur. Nulla gravida mauris vitae tristique gravida. Vestibulum condimentum ligula nisl, sit amet porttitor lorem porttitor vel.

Praesent vitae gravida mi. In non felis metus. Duis at nisi eget erat convallis tempus ut ut risus. Mauris vehicula justo ac sapien placerat, a tincidunt metus feugiat. Nullam pretium mollis dui non facilisis. Cras sit amet tortor eu erat sodales tincidunt. Aliquam erat volutpat. Nulla facilisi. Fusce a mi et ligula interdum malesuada vel eu lorem.

Cras convallis, risus id hendrerit suscipit, magna velit tristique augue, eget mollis turpis enim sit amet orci. Suspendisse sit amet auctor nulla. Etiam iaculis risus orci, ac blandit ante hendrerit et. Integer viverra, odio nec posuere tincidunt, odio ligula tempor eros, id mollis nulla elit non felis. Pellentesque fermentum ex risus, sed maximus ligula tristique in.

Nulla facilisi. Nam ac ante id ipsum tincidunt elementum. Cras ultrices consectetur augue, ac vulputate nulla malesuada id. Fusce tempor metus vel libero rhoncus, non interdum enim bibendum. Etiam laoreet justo ut tortor tempor, non tristique felis tincidunt. Ut eget orci eu nulla lobortis auctor ut ut est. Ut ac elit sit amet nisi interdum interdum. Quisque non vulputate nunc, eu sollicitudin arcu. Integer fringilla, dui id laoreet vulputate, felis sapien iaculis purus, eu viverra nunc purus sit amet erat.

